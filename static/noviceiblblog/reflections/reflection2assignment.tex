\documentclass{article}

\usepackage{amsmath,amsthm,latexsym}
\usepackage{hyperref}
\usepackage{color}

\addtolength{\topmargin}{-.875in}
\evensidemargin 0.0in
\oddsidemargin 0.0in
\textwidth 6.5in
\textheight 9in

\pagestyle{empty}

\begin{document}

\begin{tabular}{lcr}
Quincy University & \hspace{3in} & Due Mon 9/19 in class\\
Math 124 & \hspace{3in} & Reflection 2\\
Instructor: David Failing & \hspace{3in} & Fall 2016
\end{tabular}

\hrule

\vspace{\baselineskip}

The final item you will turn in is a 2+ page typed reflection (either a Word document or PDF. No ``Pages'' or other word processors, please). Turn your paper in by clicking on the ``Reflection 2 - Math Autobiography '' link under this week's Moodle section, and clicking ``Add submission'' to upload your file by 9:00am on Monday. Please title your file ``Lastname Firstname - Reflection 2,'' e.g. ``Failing David - Reflection 2.'' You must also bring a printed copy to class on Monday.

\section*{Growth Mindset}

\noindent\textbf{Purpose of the Assignment:} Have you ever said to someone ``I'm bad at math,'' or some similar statement? Is that something you would ever say about English? History? Your favorite sport or hobby? Neuroscience research shows a strong connection between the attitudes and beliefs students hold about themselves and their academic performance, and that what students believe about themselves affects how their brains approach learning. Work by several psychologists has shown that even small changes in attitude can have a major impact on performance. As your instructor, I want to help you think more clearly about your own thinking, and to empower you to make these changes early in your college career. In order to learn a bit more about mindsets, please read the article and watch the TED talk (linked below) before you begin your reflection.\\

\noindent\textbf{`I'm Not A Math Person' Is No Longer A Valid Excuse}:\\ \url{http://www.businessinsider.com/being-good-at-math-is-not-about-natural-ability-2013-11}\\
\textbf{The Power of Belief}:\\ \url{https://www.youtube.com/watch?v=Yn966v5INaI&list=PL4111402B45D10AFC&index=6}\\

\noindent\textbf{Content:} Your reflection should address the sections listed below.  Read them all before you begin writing. I've listed some questions to help guide you, but please don't just go through and answer each question separately.  Weave your answers together to tell a story. The questions are just to help get you thinking.  Remember the purpose of the paper.   The key to writing a good piece is to give lots of detail.  See the example below:\\

\noindent\textit{Not enough detail:} I hated math in fourth grade, but it got better in sixth grade.\\

\noindent\textit{Good detail:} I hated math in fourth grade because I had trouble learning my multiplication tables.  I was really slow at doing problems, and I was always the last one to finish the timed tests.  It was really embarrassing...

\section*{Section 1: Basic Definitions}
\begin{itemize}
\item How would you define fixed mindset? Illustrate your definition with an example.
\item How would you define growth mindset? Illustrate your definition with an example.
\item How would you define \textit{productive} failure? How does it differ from ``regular'' failure? Illustrate your definition with an example.
\end{itemize}

\section*{Section 2: Mindsets in Your Life}
\begin{itemize}
\item Give an example of an area where you have a fixed mindset. Why do you think you do in this area?
\item Give an example of an area where you have a growth mindset. Why do you think you do in this area? What strategies do you use in this area that could improve your areas of fixed mindset?
\item Think about the last time you had a major setback, failure or rejection in your life. Did you hear the fixed mindset voice in your head? What did it say? Now, how would you answer it with a growth mindset voice? What would that dialogue sound like?
\item Think of something about yourself you've been wanting to change. What is it? Has a fixed mindset prevented you from doing this? Think about it from a growth mindset and spell out a concrete plan for change.
\end{itemize}

\section*{Additional Resources}
If you want to learn more about the ideas in the first video and article, check out any of the following. Please reference them in your reflection if you do.\\

\noindent\textbf{ARTICLES}\\
Not A Math Person (KQED News)\\
\href{https://ww2.kqed.org/mindshift/2015/11/30/not-a-math-person-how-to-remove-obstacles-to-learning-math/}{http://tinyurl.com/GMNotMath}\\\\\
Does Teaching Kids To Get `Gritty' Help Them Get Ahead? (NPR)\\
\href{http://www.npr.org/sections/ed/2014/03/17/290089998/does-teaching-kids-to-get-gritty-help-them-get-ahead}{http://tinyurl.com/GMGrit}\\\\\
In schools, self-esteem boosting is losing favor to rigor, finer-tuned praise (Washington Post)\\
\href{http://tinyurl.com/GMEsteem}{http://tinyurl.com/GMEsteem}\\\\\
If You?re Open to Growth, You Tend to Grow (New York Times)\\
\href{http://tinyurl.com/GMOpenGrowth}{http://tinyurl.com/GMOpenGrowth}\\\\\
Even Geniuses Work Hard (Educational Leadership)\\
\href{http://www.ascd.org/publications/educational-leadership/sept10/vol68/num01/Even-Geniuses-Work-Hard.aspx}{http://tinyurl.com/GMGeniuses}\\\\\

\noindent\textbf{VIDEOS}\\
Learning: How It Works And How To Do It Better by Seth Godin (11:39)\\
\url{https://www.youtube.com/watch?v=u9WpHHJz5Dc}\\\\\
The Foundation of Teaching Growth Mindset: Unleashing the Learning Machine (19:12)\\
\url{https://www.youtube.com/watch?v=hLqT6_1cv00}\\\\\
You Can Learn Anything (1:30) \\
\url{https://www.youtube.com/watch?v=JC82Il2cjqA&list=PL4111402B45D10AFC&index=1}\\\\\
The Power of Yet, Carol Dweck Ted Talk (11:18)\\
\url{https://www.youtube.com/watch?v=J-swZaKN2Ic}\\

\newpage
\section*{Assignment Requirements}
\textbf{Format:} Double-spaced, 1-inch margins, 12-point font\\
\textbf{Minimum length:} 2 pages\\

\noindent\textbf{Your paper needs to have the following information in the upper right corner:}\\
Your name\\
Growth Mindset\\
Course number \\
Time course meets\\

\textbf{Grading:} This paper will not be graded for writing style and composition in the same way that you would be graded in a writing course, but I do expect proper spelling and grammar. If this is an issue for you, have someone proof read it for you. I will grade using the following rubric.

\section*{CONTENT}
Each section is fully covered and includes relevant detail.\\\\\
2: Complete; 1: Incomplete; 0: Missing\\ \\\
Defines the two mindsets and give an appropriate example of each. \hfill\enspace 0 \enspace\enspace 1 \enspace\enspace 2\\
Defines productive failure and distinguishes it. \hfill\enspace 0 \enspace\enspace 1 \enspace\enspace 2\\
Illustrates fixed and growth mindsets with a personal example. \hfill\enspace 0 \enspace\enspace 1 \enspace\enspace 2\\
Discusses setbacks and opportunities for growth. \hfill\enspace 0 \enspace\enspace 1 \enspace\enspace 2\\

\section*{FORMAT}
Double-spaced, correct margins, 12-point font, minimum length (2 pages) met \hfill\enspace 0 \enspace\enspace 1 {\enspace\enspace \color{white} 2}\\
Correct file name and header: \hfill\enspace 0 \enspace\enspace 1 {\enspace\enspace \color{white} 2}


%%%SUSAN CROOK'S GM ACTIVITY
%FOR THE FIRST WEEK ONLY, rather than describing a failure, please read the article below and watch the video then answer the following questions:
%
%1) What is your definition of productive failure?
%
%2) How would you define growth mindset?
%
%3) Please reflect on the article and video. What do you think of these ideas? Where do you see places to apply them to your life both inside and outside the classroom?
%
%4) What can I do to help you succeed in this class and at having a growth mindset?
%
%https://www.ted.com/.../carol_dweck_the_power_of...
%
%https://ww2.kqed.org/.../not-a-math-person-how-to-remove.../

%%%LAURIE ZACK'S GM ACTIVITY
%First Day Activity-Growth Mindset Article/Video
%Note: You can choose to do this in class/out of class to whatever degree works for you.
%Day 1: Either Assign Video(s) or Article(s) before class or for second day. With the following prompts: 
%1. What is a fixed mindset?
%2. What is a growth mindset?
%3. Given an example of an area where you have a fixed mindset? Why do you think you do in this area? Give an example of an area where you have a growth mindset? Why do you think you do in this area? 
%4. Write down something that you found interesting in this article/video
%Day 2: Have small group discussion first, having students talk with their peers in groups of 3/4, then open up the discussion to the class. Then as a class, pose these questions:
%1.  Think about the last time you had a major setback, failure or rejection in your life. Did you hear the fixed mindset voice in your head? What did it say? Now, how would you answer it with a growth mindset voice? What would that dialogue sound like?
%2. Think of something about yourself you?ve been wanting to change. What is it? Has a fixed mindset prevented you from doing this? Think about it from a growth mindset and spell out a concrete plan for change.
%Homework/To Try and Change their mindset: As Homework, send the students home with the two questions you just had in class discussion. Have them think more deeply about it and write down their answers. Writing down how they might change their mindset will have a greater impact then them discussing it in class. You can always provide more videos/articles that you don?t assign. 
%List of Articles for you to choose from to read: ( I suggest you pick perhaps two of these if you just want to have the read articles, they are all good and none are too long)
%http://www.npr.org/sections/ed/2014/03/17/290089998/does-teaching-kids-to-get-gritty-help-them-get-ahead
%http://www.ascd.org/publications/educational-leadership/sept10/vol68/num01/Even-Geniuses-Work-Hard.aspx
%http://www.wsj.com/articles/SB10001424052702303740704577522882725680396
%https://www.washingtonpost.com/local/education/in-schools-self-esteem-boosting-is-losing-favor-to-rigor-finer-tuned-praise/2012/01/11/gIQAXFnF1P_story.html
%http://www.nytimes.com/2008/07/06/business/06unbox.html?_r=2&ref=business&oref=slogin
%List of Videos to Watch (First one if my favorite if you pick just one)
%https://www.youtube.com/watch?v=u9WpHHJz5Dc (Learning: How it works and How to do it better, 10 minutes, by Seth Godin)
%https://www.youtube.com/watch?v=JC82Il2cjqA&list=PL4111402B45D10AFC&index=1 (You can Learn Anything) (1:30min) 
%https://www.youtube.com/watch?v=Yn966v5INaI&list=PL4111402B45D10AFC&index=6 (10 min, ?The Power of Belief?) 
%https://www.youtube.com/watch?v=hLqT6_1cv00 (The Foundation of Teaching Growth Mindset: Unleashing the Learning Machine, 20 min)
%https://www.youtube.com/watch?v=J-swZaKN2Ic (The Power of Yet, Carol Dweck Ted Talk, 10 min)

\end{document}