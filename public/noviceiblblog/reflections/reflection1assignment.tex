\documentclass{article}

\usepackage{amsmath,amsthm,latexsym}
\usepackage{hyperref}
\usepackage{color}

\addtolength{\topmargin}{-.875in}
\evensidemargin 0.0in
\oddsidemargin 0.0in
\textwidth 6.5in
\textheight 9in

\pagestyle{empty}

\begin{document}

\begin{tabular}{lcr}
Quincy University & \hspace{3in} & Due Fri 8/26 in class\\
Math 124 & \hspace{3in} & Reflection 1\\
Instructor: David Failing & \hspace{3in} & Fall 2016
\end{tabular}

\hrule

\vspace{\baselineskip}

The final item you will turn in is a 2+ page typed reflection (either a Word document or PDF. No ``Pages'' or other word processors, please). Turn your paper in by clicking on the ``Reflection 1 - Math Autobiography '' link under this week's Moodle section, and clicking ``Add submission'' to upload your file by 9:00am on Friday. Please title your file ``Lastname Firstname - Reflection 1,'' e.g. ``Failing David - Reflection 1.'' You must also bring a printed copy to class on Friday, with this sheet stapled as the last page, in order to receive a grade.

\section*{Math Autobiography}

\noindent\textbf{Purpose of the Assignment:} As your instructor, I want to get to know you as a person and as a student of mathematics, and to help you identify what you believe about math.  This will also help me better create the course \textit{for you}.  It also guides the university as we work to improve our services to students.\\

\noindent\textbf{Content:} Your autobiography should address the four sections listed below.  Read them all before you begin writing. I've listed some questions to help guide you, but please don't just go through and answer each question separately.  The questions are just to help get you thinking.  Remember the purpose of the paper.  Write about the things that will give me a picture of you.  The key to writing a good piece is to give lots of detail.  See the example below:\\

\noindent\textit{Not enough detail:} I hated math in fourth grade, but it got better in sixth grade.\\

\noindent\textit{Good detail:} I hated math in fourth grade because I had trouble learning my multiplication tables.  I was really slow at doing problems, and I was always the last one to finish the timed tests.  It was really embarrassing...

\section*{Section 1: Introduction}
\begin{itemize}
\item How would you describe yourself?
\item Where are you from? How did you decide to attend Quincy University?
\item What is your educational background? Did you just graduate from high school? Are you a freshman/sophomore/junior/senior? Have you been out of school for a few years? If so, what have you been doing since then?
\item General interests: favorite subjects in school, favorite activities or hobbies.
\end{itemize}

\section*{Section 2: Experience with Math}
\begin{itemize}
\item What math classes have you taken and when? Which ones did you like or dislike, and why? What about your instructors?
\item What have your experiences in math classes been like? What were your grades like? What contributed to your success or failure in those classes?
\item How do you feel about math? Has that changed over time? How would you describe your ability in math? What does it mean to be ``good at math?''
\item How would you define mathematics to someone who doesn't know what it is?
\item In what ways have you used math outside of school?
\end{itemize}

\section*{Section 3: Learning Styles and Habits (Specifically for Math)}
\begin{itemize}
\item Do you learn best from reading, listening or doing?
\item Do you prefer to work alone or in groups?
\item What do you do when you get ``stuck''?
\item Do you ask for help?  From whom?
\item Describe some of your study habits. For example: Do you take notes? Are they helpful? Are you organized? Do you procrastinate? Do you read the textbook?
\end{itemize}

\section*{Section 4: The Future}
\begin{itemize}
\item What are your expectations for this course? How do you think you will do in this course?
\item What are your anxieties about taking this course?
\item What are your responsibilities as a student in this course? What do you expect from your instructor?
\item What are your educational and life goals?
\item What is one nonacademic obstacle that might affect these goals, this semester?
\item How does this course fit into your educational goals?
\end{itemize}

\section*{Assignment Requirements}
\textbf{Format:} Double-spaced, 1-inch margins, 12-point font\\
\textbf{Minimum length:} 2 pages\\

\noindent\textbf{Your paper needs to have the following information in the upper right corner:}\\
Your name\\
Math Autobiography\\
Course number \\
Time course meets\\

\textbf{Grading:} This paper will not be graded for writing style and composition in the same way that you would be graded in a writing course, but I do expect proper spelling and grammar. If this is an issue for you, have someone proof read it for you. I will grade using the following rubric.

\section*{CONTENT}
Each section is fully covered and includes relevant detail.\\
2: Complete; 1: Incomplete; 0: Missing\\ 
Section 1: Introduction\hfill\enspace 0 \enspace\enspace 1 \enspace\enspace 2\\
Section 2: Experience With Math\hfill\enspace 0 \enspace\enspace 1 \enspace\enspace 2\\
Section 3: Learning Styles and Habits\hfill\enspace 0 \enspace\enspace 1 \enspace\enspace 2\\
Section 4: The Future\hfill\enspace 0 \enspace\enspace 1 \enspace\enspace 2\\

\section*{FORMAT}
Double-spaced, correct margins, 12-point font: \hfill\enspace 0 \enspace\enspace 1 {\enspace\enspace \color{white} 2}\\
Minimum length (2 pages) met: \hfill\enspace 0 \enspace\enspace 1 {\enspace\enspace \color{white} 2}

\end{document}